\documentclass[12pt,letterpaper]{article}
\usepackage{natbib}

%Packages
\usepackage{pdflscape}
\usepackage{fixltx2e}
\usepackage{textcomp}
\usepackage{fullpage}
\usepackage{float}
\usepackage{latexsym}
\usepackage{url}
\usepackage{epsfig}
\usepackage{graphicx}
\usepackage{amssymb}
\usepackage{amsmath}
\usepackage{bm}
\usepackage{array}
\usepackage[version=3]{mhchem}
\usepackage{ifthen}
\usepackage{caption}
\usepackage{hyperref}
\usepackage{amsthm}
\usepackage{amstext}
\usepackage{enumerate}
\usepackage[osf]{mathpazo}
\usepackage{dcolumn}
\usepackage{lineno}
\usepackage{dcolumn}
\newcolumntype{d}[1]{D{.}{.}{#1}}

\pagenumbering{arabic}


%Pagination style and stuff
\linespread{2}
\raggedright
\setlength{\parindent}{0.5in}
\setcounter{secnumdepth}{0} 
\renewcommand{\section}[1]{%
\bigskip
\begin{center}
\begin{Large}
\normalfont\scshape #1
\medskip
\end{Large}
\end{center}}
\renewcommand{\subsection}[1]{%
\bigskip
\begin{center}
\begin{large}
\normalfont\itshape #1
\end{large}
\end{center}}
\renewcommand{\subsubsection}[1]{%
\vspace{2ex}
\noindent
\textit{#1.}---}
\renewcommand{\tableofcontents}{}
%\bibpunct{(}{)}{;}{a}{}{,}

%---------------------------------------------
%
%       START
%
%---------------------------------------------

\begin{document}
\begin{center}
\noindent{\Large \bf Timespines: does predator-prey arms race interaction drives morphological adaptation through time?}
\bigskip

\noindent {\normalsize \sc Thomas Guillerme$^{1,*}$,
Kevin Healy$^{2}$,
Adam Kane$^{3}$,
Marco Castiello$^{4}$,
Richard Dearden$^{4}$,
Rafael Delcourt$^{5}$,
Anna Jerve$^{4}$,
Martin Brazeau$^{4}$,
Vera Weisbecker$^{1}$,
and Lauren Sallan$^{6}$}\\

\noindent {\small \it 
$^1$School of Biological Sciences, University of Queensland, St. Lucia, Queensland, Australia.\\
$^2$School of Biology, St Andrews University, St Andrews, United Kingdom;\\
$^3$School of Biological, Earth and Environmental Sciences, University College Cork, Cork, Ireland;\\
$^4$Department of Life Sciences, Imperial College London, Silwood Park Campus, United Kingdom;\\
$^5$Zoology Department, Trinity College Dublin, Dublin, Ireland;\\
$^6$Earth \& Environmental Science, University of Pennsylvania, Philadelphia, USA.\\}

\end{center}

\begin{abstract}
Body armour is a consistent trait observed in animals throughout the whole fossil record.
This is due, in part because of it's taphonomy (i.e. being relatively well preserved compared soft tissue characters) but also to it's ecological importance.
In fact, morphological diversity have been hypothesised to rise at a really fast rate in part due to the role of body armour in the predator prey evolutionary arms race.
Under this hypothesis, we expect the evolution of body armour in animals to be tightly correlated to predator's morphological evolution.
More specifically, we expect that preys that are too small or to big relatively to an ecosystem's dominant predator will be less likely to evolve body armour.
Conversely, preys that fall in the ``danger zone'', (i.e. falling in the size range of their dominant predator) will be more likely to evolve body armour.

Here we estimate the body size at the appearance or disappearance of armour traits (spines or plates) of all fishes families ranging from the Devonian (419 million years ago) to the present to test whether the appearance of armour is correlated to prey's size relative to their top predator's size.
We found evidences for a relation between relative prey size and the appearance of armour traits.
These results suggest that ecological relationship in animals can be a major driver of morphological evolution through time.
\end{abstract}


\end{document}